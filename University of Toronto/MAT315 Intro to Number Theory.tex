\documentclass[12pt]{article}
\usepackage{amsmath}
\usepackage{amsfonts}
\usepackage{amsthm}
\usepackage{amssymb}
\usepackage{float}
\usepackage[margin=2.5cm]{geometry}

\title{MAT315 (Summer 2022) Notes}
\author{Zhongmang Cheng}
\date{\today}

\begin{document}
\maketitle

% ==========Fundamentals==========
\section*{Fundamentals}
\textbf{Definition (divides):} $a\mid b$ if and only if $b=k\cdot a$ for some integer k.\\
$a\mid 0, a\mid a, a\mid -a, 1\mid a$ for every $a$.\\
\\
\textbf{Definition (fully divides):} $p^e\mid\mid a$ if $p^e\mid a$ but $p^{e+1}\nmid a$.\\
$p^\alpha\mid \mid a$ and $p^\beta\mid \mid b \implies p^{\alpha+\beta}\mid\mid ab$, $p^{\alpha-\beta}\mid \mid \frac{a}{b}$\\
\\
\textbf{Division Algorithm:} If $b\neq 0$, then there are unique integers $k$ and $r$ such that:\\$a=k\cdot b+r$ and $0\leq r <|b|$.\\
\\
\textbf{Fundamental Theorem of Arithmetic:} Every $n\geq 2$ has a prime factorization $n=p_1^{a_1}p_2^{a_2}\cdots p_k^{a_k}$ where $p_i$ are distinct primes and $a_i$ are positive integers. This factorization is unique up to re-ordering.\\
\\
\textbf{Dirichlet's Theorem:} There are infinitely many primes of the form $ak+b$ if and only if $(a,b)=1$.\\
\\
\textbf{Diophantine Equation:}
$ax+by=c$ has solution if and only if $(a,b)\mid c$.\\
The set of all solution is:
\begin{center}
    $\{x\in\mathbb{Z}:x_0+t\cdot\frac{m}{(m,a)}\}$
    or $\{x\in\mathbb{Z}:x= x_0\pmod {\frac{m}{(m,a)}}\}$\\
\end{center}


% ==========Modular Arithmetic's==========
\section*{Modular Arithmetic's}
If $a\equiv c\pmod n$ and $b\equiv d\pmod n$, then:
\begin{itemize}
    \setlength\itemsep{0em}
    \item $a+b\equiv c+d\pmod n$
    \item $ab\equiv cd\pmod n$
    \item $a^k\equiv c^k\pmod n$ where $k\in\mathbb{N}$
\end{itemize}
Suppose $d\geq 1$ and $d\mid m$, then $a\equiv b\pmod m\implies a\equiv b\pmod d$\\
Suppose $c\geq 0$, then $a\equiv b\pmod m\implies ac\equiv bc\pmod {mc}$\\
$ax\equiv ay\pmod m\implies x\equiv y\pmod {\frac{m}{(m,a)}}$\\


% ==========CRT and related conclusions==========
\section*{CRT and related conclusions}
\textbf{Chinese Remainder Theorem:} $x\equiv  a_1\pmod m_1, x\equiv  a_2\pmod m_2\cdots x\equiv  a_k\pmod m_k$ with $(m_i,m_j)= 1$ for all $i\neq j$ has a unique solution $x\equiv  a\pmod {m_1m_2\cdots m_k}$ in $\mathbb{Z}_{m_1m_2\cdots m_k}$ for some $a$.\\
\\
$x\equiv a \pmod {m_1m_2} \implies x\equiv a\pmod {m_1}, x\equiv a\pmod {m_2}$\\
$x\equiv a\pmod {m_1}, x\equiv a\pmod {m_2} \implies x\equiv a\pmod {[m_1,m_2]}$\\
\\
$x\equiv a_1\pmod {m_1}, x\equiv a_2\pmod {m_2} \implies$ $x$ has 0 or 1 solution in $\mathbb{Z}_{[m_1,m_2]}$\\
$x\equiv a_1\pmod {m_1}, x\equiv a_2\pmod {m_2}, (m_1,m_2)= 1 \implies$ $x$ has unique solution in $\mathbb{Z}_{m_1m_2}$\\
\\
$x$ has $n_1$ possible values mod $m_1$, $n_2$ possible values mod $m_2$, $(m_1, m_2)= 1$\\
$\implies$ x has $n_1n_2$ possible values in $\mathbb{Z}_{m_1m_2}$


% ==========FLT and related applications==========
\section*{FLT and related applications}
\textbf{Fermat Little Theorem:} if $a$ is not divisible by $p$, then $a^{p-1}\equiv 1\pmod p$.\\
$a^{p-1}\equiv 1\pmod p$ for $(a,p)= 1$ where $p$ is a prime.\\
$a^p\equiv a\pmod p$ for all $a$ where $p$ is a prime.\\
\\
\textbf{Wilson's Theorem:} $n\geq 2$ is a prime if and only if $(n-1)!\equiv  -1\pmod n$.\\
\\
\textbf{Primality Test:} “$n$ passes base $a$ test” if $a^n\equiv a\pmod n$\\
\\
\textbf{Definition (pseudo-prime):} a composite number $n$ such that $2^n\equiv  2\pmod n$.\\
If $p$ is an odd prime, then $x^2+1\equiv  0\pmod p$ has a solution if and only if $p\equiv  1\pmod 4$.


% ==========Polynomials==========
\section*{Polynomials}
Let $p(x)$ be a polynomial with integer coefficients, then:\\
$a\equiv  b\pmod n \implies p(a)\equiv  p(b)\pmod n$\\
\\
\textbf{Lagrange Theorem:} $f(x)=a_dx^d+a_{d-1}x^{d-1}\cdots a_1x+a_0$ is a polynomial with integer coefficients such that $a_i\not\equiv 0\pmod p$ for at least one $i$. Then, $f(x)\equiv  0\pmod p$ has at most $d$ solutions mod p. (If $f(x)=a_dx^d+a_{d-1}x^{d-1}\cdots a_1x+a_0$ has more than $d$ solutions mod p, then $a_i\equiv 0\pmod p$ for all $i$.)\\
\\
\textbf{Remark:} $f(a)\equiv 0\pmod p \implies f(x)\equiv (x-a)g(x)\pmod p$\\
\\
$x^a-1=(x-1)(x^{a-1}+x^{a-2}\cdots+x+1)$\\
$x^{2a+1}+1=(x+1)(x^{2a}-x^{2a-1}+x^{2a-2}\cdots-x+1)$\\
If $2^m+1$ is prime, then $m=2^n$ for some $n$\\
If $2^m-1$ is prime, then $m$ must be prime\\
\\
\textbf{Hensel's Lemma:} Suppose $f(a)=0\pmod p$, then:
\begin{itemize}
    \setlength\itemsep{0em}
    \item if $f'(a)\neq 0\pmod p$, then $a$ can be lifted uniquely to $p^2$.
    \item if $f'(a)\equiv 0\pmod p$ and $\frac{f(a)}{p}\neq 0\pmod p$, then $a$ cannot be lifted uniquely to $p^2$.
    \item if $f'(a)\equiv 0\pmod p$ and $\frac{f(a)}{p}\equiv  0\pmod p$, then $a$ can be lifted to $p$ solutions in $p^2$.
\end{itemize}


% ==========Euler Function and Primitive Roots==========
\section*{Euler Function and Primitive Roots}
\textbf{Definition (unit):} $u$ is a unit mod n if it has an inverse.\\
$u$ has an inverse $u^{-1}$ such that $u\cdot u^{-1}\equiv  1\pmod n$ only if $(u,n)=1$.\\
\\
\textbf{Definition (Euler Function):} $\phi(n)$ represent the number of units in $\mathbb{Z}_n$.\\
\\
\textbf{Euler's Theorem:} Suppose $(a,n)=1$, then $a^{\phi(n)}\equiv 1\pmod n$.\\
\\
$\phi(p_1^{a_1}p_2^{a_2}\cdots p_k^{a_k})=\phi(p_1^{a_1})\phi(p_2^{a_2})\cdots \phi(p_k^{a_k})$ and $\phi(p^k)=p^k\cdot (1-\frac{1}{p})$ for $p$ prime, $k\geq 1$.\\
$\phi(mn)=\phi(m)\phi(n)$ for $(m,n)=1$.\\
\\
\textbf{Definition (primitive root):}\\ $g$ is a primitive root if $\{1,2\cdots,p-1\}\equiv\{g,g^2\cdots,g^{p-1}\}$ in $\mathbb{Z}_p$ (generator for $(\mathbb{Z}_n)^\times$).\\
\\
\textbf{Definition (order):}\\ The smallest postive integer $k$ such that $u^k\equiv 1\pmod n$, denoted by $ord_n(u)$.\\
\\
\textbf{Remark:} $u^k\equiv 1\pmod n \iff ord_n(u)\mid k$\\
\\
A unit u is a primitive root (generates the complete set of $(\mathbb{Z}_p)^\times) \iff ord_n(u)=\phi(p)$.\\
If $g$ is a primitive root modulo p, then $g^k\equiv 1\pmod p \iff p-1\mid \phi(p)$.\\
If $g$ is a primitive root modulo p, then $ord_n(g^a)=\frac{\phi(n)}{\phi(n),a)}$\\
\\
Let $d\mid p-1$ be positive, then there are exactly $\phi(d)$ units mod p of order $d$.
\\
The sum of all $\phi(m)$ such that $m\mid n $ equals to $n$\\
\\
There are $\phi(d)$ units of order $d$ mod p when $d\mid p-1$.\\
\\
\textbf{Existence and related lemmas:} $Z_n$ has a primitive root if and only if $n=1,2,4$ or $n=p^m$ or $n=2\cdot p^m$ where $p\neq 2$ is prime.
\begin{itemize}
    \item Let $m\geq 2$, if $g$ is a primitive root mod $p^m$, then $g$ is a primitive root mod $p^{m+1}$.
    \item Let $n$ be odd, if $g$ is a primitive root mod $n$ and $g$ is odd, then $g$ is a primitive root mod $2n$.
    \item If $n=a\cdot b$ with $(a,b)=1$ and $a,b>2$, then $\mathbb{Z}_n$ has no primitive root.
\end{itemize}
\textbf{Theorem:} $ord_{2^n}(5)=2^{n-2}$\\
\textbf{Theorem:} Units of $\mathbb{Z}_{2^n}$ can be generated by two units: 5 and -1.


% ==========Applications in Cryptography==========
\section*{Applications in Cryptography}
In the following context, we assume that $x$ is the message we would like to encode, and $m$ is the message we would like to decode. Also, we assume that finding the modulo inverse is an easy computation using Euclidean Algorithm.

\subsection*{Modular Exponential Cipher}
Public information: $p$ large prime\\
Secret information: $e$ where $(e,p-1)=1$ (Here $e$ is the key)\\
\\
To encode: compute $m\equiv x^e\pmod p$, send $m$.\\
To decode: find the inverse $f$ such that $e\cdot f\equiv 1\pmod {p-1}$, computes $m^f\pmod p$.\\
\\
\textbf{Remark:} $m^f\equiv (x^e)^f\equiv x^{ef}\equiv x^{(p-1)k+1}\equiv x\pmod p$

\subsection*{Diffie-Hellman Key Exchange}
Public information: $p$ large prime, $g$ such that $1<g<p$, and $g^a$, $g^b$\\
Secret information: $a$ only known by A, $b$ only known by B (Here $g^{ab}$ is the key)\\
\\
A computes $g^a\pmod p$ and send it to B. B computes $g^b\pmod p$ and send it to A. A and B compute $(g^b)^a\pmod p$ and $(g^a)^b\pmod p$, where $g^{ab}$ is the key. Then encode and decode as the previous method using the key.

\subsection*{RSA Public Key}
Public information: $e$ such that $(e,(p-1)(q-1))=1$, $N=p\cdot q$\\
Secret information: $p$ and $q$ large prime\\
\\
To encode: compute $m\equiv x^e\pmod {pq}$, send $m$.\\
To decode: find the inverse $f$ such that $e\cdot f\equiv 1\pmod {(p-1)(q-1)}$, computes $m^f\pmod {pq}$.\\
\\
\textbf{Remark:} $m^f\equiv (x^e)^f\equiv x^{ef}\equiv x^{(p-1)(q-1)k+1}\equiv x^{\phi(pq)k+1}\equiv x\pmod p$


% ==========Applications in Cryptography==========
\section*{Quadratic Residue}
\textbf{Definition (Quadratic Residue):} $n$ is a positive integer, $a$ is a unit in $\mathbb{Z}_n$.\\
If $x^2\equiv a\pmod n$ has a solution, then it's a quadratic residue (QR). Otherwise, it's a quadratic non-residue (QNR).\\
\\
\textbf{Legendre Symbol and related properties}\\
Let $p$ be an odd prime. $(\frac{a}{p})=1$ if $a$ is QR, $(\frac{a}{p})=-1$ if $a$ is NQR, $(\frac{a}{p})=0$ if $a=0\pmod p$.
\begin{itemize}
    \item $(\frac{1}{p})=1$, $(\frac{a}{p})=(\frac{a^{-1}}{p})$, $(\frac{ab}{p})=(\frac{a}{p})\cdot(\frac{b}{p})$
    \item If $g$ is a primitive root, then $(\frac{g^k}{p})=(-1)^k$.
    \item If $a$ is a unit, then $(\frac{a^2}{p})=1$, $(\frac{a^2b}{p})=(\frac{b}{p})$.
    \item There are $\frac{p-1}{2}$ QR, and $\frac{p-1}{2}$ QNR.
\end{itemize}
$p\equiv 1\pmod 4 \implies$ -1 is a QR\\
$p\equiv 3\pmod 4 \implies$ -1 is a QNR\\
\\
\textbf{Euler's Criterion:} $(\frac{a}{p})\equiv a^{\frac{p-1}{2}}\pmod p$\\
\\
\textbf{Gauss' Lemma and conclusions} Suppose $a$ is a unit mod $p$, write each of $a, 2a \cdots, \frac{p-1}{2}a$ between $-\frac{p-1}{2}$ and $\frac{p-1}{2}$ mod p. Let $n$ be the number of negative signs. Then $(\frac{a}{p})=(-1)^n$.\\
\\
$(\frac{2}{p})=1$ if $p\equiv 1, 7\pmod 8$\\
$(\frac{2}{p})=-1$ if $p\equiv 3, 5\pmod 8$\\
\\
$(\frac{3}{p})=1$ if $p=1, 11\pmod {12}$\\
$(\frac{3}{p})=-1$ if $p=5, 7\pmod {12}$\\
\\
\textbf{Law of Quadratic Reciprocity:} Suppose $p\neq q$ are odd primes, then:
\begin{center}  
    $(\frac{p}{q})=(\frac{q}{p})\cdot (-1)^{\frac{p-1}{2}\cdot \frac{q-1}{2}}$.
\end{center}
Let $k\geq 3$, then $a$ is a QR mod $2^k$ if and only if $a\equiv 1\pmod 8$.


\end{document}
